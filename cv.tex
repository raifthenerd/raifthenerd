% !TEX encoding = UTF-8 Unicode
% !TEX TS-program = lualatex

\PassOptionsToPackage{pdfencoding=auto}{hyperref}
\documentclass[a4paper,10pt,nolroman]{moderncv}
\moderncvtheme{banking}
\moderncvcolor{black}
\nopagenumbers{}

\usepackage[scale=0.8]{geometry}

\usepackage[type1]{newtxtext}
\usepackage[T1]{fontenc}

\usepackage[english]{babel}

\name{Seokjin}{Han}
\homepage{https://blog.raifthenerd.com}
\social[linkedin]{raifthenerd}
\social[github]{raifthenerd}

\begin{document}
\makecvtitle%
\section{Education}
\cventry{Mar 2017 --- Feb 2019}{M.Sc.\ in Statistics}{Seoul National University}{Seoul, Korea}{}{Supervisor: Prof.\ Jaeyong~Lee @ Bayesian Statistics Lab.\newline
Thesis: \emph{De Novo} Drug Design Using Deep Generative Models}
\cventry{Mar 2012 --- Feb 2017}{B.Sc.\ in Industrial Engineering \& B.Sc. in Statistics}{Seoul National University}{Seoul, Korea}{Cum Laude}{Minor in Computer Science and Engineering.}
\cventry{Mar 2009 --- Feb 2012}{High School Diploma}{Seoul Science High School}{Seoul, Korea}{}{}
\section{Experience}
\subsection{Vocational}
\cventry{Dec 2023 --- Current}{ML Research Scientist}{Twelve Labs}{Seoul, Korea}{ML Modeling Team}{}
\cventry{Jan 2019 --- Dec 2023}{AI Scientist}{Standigm}{Seoul, Korea}{AI-Bio Team}{Developed core models of Standigm ASK\texttrademark, the in-house platform that utilizing various types of biological data (including knowledge graphs and NGS data) with neural networks.}
\cventry{Dec 2015 --- Feb 2016}{Research Intern}{Bayesian Statistics Lab @ Seoul National University}{Seoul, Korea}{}{Implemented various stochastic search algorithms for variable selection in Bayesian linear models.}
\cventry{Jul 2015 --- Aug 2015}{Machine Learning Engineer}{PRND Company}{Seoul, Korea}{}{Developed a market price prediction service for used cars using Bayesian methods.}
\subsection{Teaching}
\cventry{Fall 2018}{Teaching Assistant}{Statistical Computing and Lab. (326.212)}{Seoul National University}{}{Instructor: Prof. Joong-Ho~(Johann)~Won}
\cventry{Spring 2018}{Teaching Assistant}{Statistics Concept and Lab. (033.021)}{Seoul National University}{}{Instructor: Prof. Sinsub~Cho}
\cventry{Fall 2017}{Teaching Assistant}{Data Mining Methods and Lab. (326.413)}{Seoul National University}{}{Instructor: Prof. Yongdai~Kim}
\section{Skills}
\subsection{Computer}
\cvdoubleitem{Programming Languages}{Julia, Rust, Python, R}{Deep Learning Framework}{Flux.jl, PyTorch}
\subsection{Languages}
\cvdoubleitem{English}{Professional working proficiency.}{Korean}{Native proficiency.}
\closesection{}\pagebreak
\section{Publications}
$^\dag$: equal contributions, $^\ast$: corresponding author(s).
\subsection{Papers}
\cvlistitem{\textbf{Seokjin~Han}$^{\dag\ast}$, Ji~Eun~Lee$^{\dag}$, Seolhee~Kang, Minyoung~So, Jin~Hee, Jang~Ho~Lee, Sunghyeob~Baek, Hyungjin~Jun, Tae~Yong~Kim and Yun-Sil~Lee$^{\ast}$ (2024).
  Standigm ASK\texttrademark: Knowledge Graph and Artificial Intelligence Platform Applied to Target Discovery in Idiopathic Pulmonary Fibrosis.
  \emph{Briefings in Bioinformatics,} \emph{25}(2). doi: \href{https://doi.org/10.1093/bib/bbae035}{10.1093/bib/bbae035}}
\cvlistitem{\textbf{Seokjin~Han}, Jinhee~Hong, So~Jeong~Yun, Hee~Jung~Koo$^{\ast}$ and Tae~Yong~Kim$^{\ast}$ (2023).
  PWN:\ enhanced random walk on a warped network for disease target prioritization.
  \emph{BMC Bioinformatics,} \emph{24}(1). doi: \href{https://doi.org/10.1186/s12859-023-05227-x}{10.1186/s12859-023-05227-x}}
\subsection{Preprint}
\cvlistitem{Dongin~Kim$^{\dag\ast}$, \textbf{Seokjin~Han}$^{\dag}$, Seong-Hyeuk~Nam and Tae~Yong~Kim (2024).
  Learning the Relationship Between Variants, Metabolic Fluxes and Phenotypes.
\emph{bioRxiv}. doi: \href{https://doi.org/10.1101/2024.03.04.577140}{10.1101/2024.03.04.577140}}
\subsection{Patents}
\cvlistitem{\textbf{Seokjin~Han}, Tae~Yong~Kim, Hee~Jung~Koo, So~Jeong~Yun (2024).
  Method and System for Searching Target Node Related to Queried Entity in Network.
  \emph{U.S. Patent Application No. 18/274,416}.}
\cvlistitem{Hee~Jung~Koo, \textbf{Seokjin~Han}, Chiwon~Son, Jang~Ho~Lee, Tae~Yong~Kim, Chanung~Jeong, Jinhan~Kim, Sang~Ok~Song, So~Jeong~Yun (2022).
  Method of predicting disease, gene or protein related to queried entity and prediction system built by using the same.
  \emph{U.S. Patent Application No. 17/297,352}.}
\section{Scholarships}
\cventry{Fall 2018}{Merit-based Scholarship}{}{}{}{}
\cventry{Spring 2018}{Merit-based Scholarship}{}{}{}{}
\cventry{Fall 2017}{Brain Korea 21 Plus}{}{}{}{}
\cventry{Fall 2015}{National Scholarship For Science and Engineering}{}{}{}{}
\cventry{Spring 2015}{The Education and Research Foundation College of Engineering SNU Scholarship}{}{}{}{}
\cventry{Fall 2014}{SNU Development Fund Scholarship}{}{}{}{}
\cventry{Spring 2014}{SNU Development Fund Scholarship}{}{}{}{}
\cventry{Fall 2013}{Merit-based Scholarship}{}{}{}{}
\cventry{Fall 2012}{SNU Development Fund Scholarship}{}{}{}{}
\cventry{Spring 2012}{Superior Academic Performance}{}{}{}{}
\section{Awards}
\cventry{Oct 28, 2016}{Korea Electronics Association \& Samsung Electronics}{Honorable Mention @ IoT Innovation Challenge}{}{}{}
\cventry{Nov 22, 2014}{LG CNS}{Second Prize @ LG CNS Hacker Camp}{}{}{}
\section{Certification}
\cventry{Nov 18, 2016}{Human Resources Development Service of Korea}{Engineer Information Processing}{}{}{}
\end{document}
